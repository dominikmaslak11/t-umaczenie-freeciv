\documentclass{article}
\usepackage{graphicx} % Required for inserting images
\usepackage{polski}
%\usepackage[polish]{babel}
\usepackage[utf8]{inputenc}
\usepackage{amsmath}
\usepackage{comment}
\usepackage{cancel}

\title{freeciv matematyka walki }
\author{Dominik Maślak}


\begin{document}

\maketitle

\section{Prawdopodobieństwo wygrania walki}

Szanse na wygranie walki są czasami intuicyjne łatwe do policzenia. Np co byś powiedział na szansę wygrania w poniższej sytuacji (biorąc pod uwagę zalety terenu i inne modyfikatory): masz 2 punkty ataku przeciwko 3 obrony i taką samą ilość punktów obrażeń? Rozważając, prawdopodobieństwo wygrania każdego starcia wynosi  $\frac{2}{5}$, możesz myśleć, że to da ci sensowne szanse na zwycięstwo coś ala 40\%. Przegrana. Masz jedynie 19\% szans na zwycięstwo. Fakt trwania bitwy przez kilka rund zmierza do tego
%\footnote{The fact that the battle lasts several rounds tends to give a larger advantage to the biggest power than one could think beforehand.} 
by dać ci większe korzyści by zwęszyć twoją moc, niż mogłeś o tym pomyśleć wcześniej.
\newline
Zakładając, że czytelnik jest zaznajomiony z zasadami walki ( dla każdej rundy brana jest przypadkowa liczba, decydująca kto wygra tą rundę, itd...), spróbujmy znaleźć matematyczną formułę reprezentującą szansę na zwycięstwo.

\section{Zapis i podstawowe obliczenia}
Weź pod uwagę, że jednostka ma siłę natarcia (wszystkie modyfikację, pola miasta, doświadczenie jednostki w walce, itd... zostaną zastosowane) $s$, a jednostka wroga ma obronę $r$. Ilość punktów obrażeń broniącej się jednostki jest zapisana jako $d_{hp}$, obrażenia zadane w zwyciężonej rundzie przez jednostkę atakującą to $a_{fp}$.
\newline
Oznaczmy jako $k$ liczbę zwycięskich rund wymaganą do wygrania walki. Jako $l$ oznaczymy liczbę przegranych rund która powoduję pokonanie ( jest to odliczanie do przegranej). Gdy jednostka atakująca powoduję zniszczenie w zwycięskiej rundzie $( a_{fp} = 1)$, $k$ jest liczbą punktów obrażeń pokonywanej jednostki $(d_{hp})$. Bardziej ogólnie,
\begin{equation*}
    k =  
   \left\lfloor \frac{d_{hp} + a{fp - 1} }{a_{fp}} \right\rfloor = 
   \left\lceil \frac{d_{hp}}{a_{fp}} \right\rceil
\end{equation*}.

mogą być policzone podobnie. Szansę wygrania każdej rundy oznaczmy jako $p$, to jest
\begin{equation*} 
    p =  \frac{s}{s+r}
\end{equation*}.
Zatem, szansa przegrania rundy to
\begin{equation*}
    q = 1-p
\end{equation*}
Oznaczmy jako $m$ maksymalną możliwą liczbę rund w których trwała walka. Zazwyczaj wynikiem każdej rundy jest to, że jakieś jednostki tracą punkty, a po pokonaniu jednostki, nie może ona więcej uderzać, jej przeciwnik musi zachować co najmniej jeden punkt, całkowita liczba możliwych przegranych to $k+l$, $m$ jest równe
\begin{equation*}
    m_d = k + l - 1
\end{equation*}
(wariant w którym się to zdarza: atakujący przegrywa $l-1$ rund, a potem wygrywa do zwycięstwa). Ale od wersji 3.0 FREECIV ta liczba może być ograniczona przez zestaw zasad, więc jeżeli $m<k+l-1$, jest tu pewne prawdopodobieństwo, że obydwie jednostki przetrwają bitwę (ograniczenie większe, bądź równe $m_d$ nic nie zmienia).

\section{Wyprowadzenie wzorów matematycznych}
Załóżmy, że walka zabierze $n$ rund do rozstrzygnięcia, obliczmy zatem prawdopodobieństwo zwycięstwa. Weź pod uwagę, że nie znamy $n$ z góry, ale załóżmy teraz że to jest podane. Dla podanego $n$, prawdopodobieństwo zwycięstwa jest prawdopodobieństwem tego, że $k$ rund będzie wygrane względem tych $n$ prób i mniej niż $l$ rund będzie przegranych. Jeżeli $n<k+l$, wówczas prawdopodobieństwo zwycięstwa jest po prostu takie, że $k$ rund będzie wygranych podczas tych $n$ prób (oznacza to, że mniej niż $l$ rund zostało przegranych). W przypadku zwycięstwa, ostatnia runda musi być ostatecznym uderzeniem (w przeciwnym razie walka powinna zostać przerwana wcześniej). Oprócz tego nie może być $k-1$ uderzeń podczas $n-1$ innych rund. Prawdopodobieństwo jest podane jako rozkład dwumianowy $B(n-1,p)$ i jest to
\begin{equation*}
\binom{n - 1}{k - 1}p^{k-1}q^{n-1}
\end{equation*}.
Jako że ostatnia runda musi zostać wygrana, mnożymy przez $p$ i uzyskujemy prawdopodobieństwo zwycięstwa po $n$ rundach
\begin{equation*}
    P_{Wn} = \binom{n-1}{k-1}p^{k}q^{n-k}
\end{equation*}.
Dla jednostki atakującej by wygrać walkę, musimy mieć (biorąc pod uwagę, że $m = m_d$)
\begin{equation}
    k \leq k + l
    \label{eq:1}
\end{equation}.
\newline
Zatem ostatecznie prawdopodobieństwo zwycięstwa wynosi:\newline
\begin{equation*}
    P_W =  \sum^{k+l-1}_1 \binom{n-1}{k-1}p^{k}q^{n-k} 
\end{equation*}.


\section{Matematyczne wyprowadzenie licznika pokonania}
Kod źródłowy freeciv używa nieco innej argumentacji \linebreak  zobacz w ./common/combat.c:unit\_win\_chance()
i \linebreak ./common/combat.c:win\_chance(). Użyjmy notacji $d$ jako liczby przegranych rund, która musi być mniejsza niż $l$ dla zwycięskiego ataku. Wówczas mamy poniższe prawdopodobieństwo zwycięstwa:
\begin{equation*}
    P_W = \sum^{l-1}_{d=0}
    \binom{k-1+d}{d}
    p^{k-1}
    q^{d}
    p
\end{equation*}
Otrzymamy \eqref{eq:1} poprzez podstawienie
\begin{equation*}
    n = k + d
\end{equation*}
i 
\begin{equation*}
    \binom{n-1}{n-k} = \binom{n-1}{k-1}
\end{equation*}.
W kodzie,\newline
$k$ to def\_N\_lose,\newline
$l$ to att\_N\_lose,\newline
$p$ to def\_P\_lose1,\newline
$(1-p)$ to att\_P\_lose1,\newline
$d$ to lr.\newline
Ogranicznik round nie jest obecnie brany pod uwagę w kodzie, ale prawdopodobnie któregoś dnia będzie.

\section{Ogólne wyprowadzenia}
Wyobraźmy sobie sytuację, w której walka zawsze składa się z $m$ rund nawet gdy zazwyczaj jedna ze stron w rzeczywistości jest zabita przed osiągnięciem liczby rund; mamy teraz ciąg $2^m$ różnych wymyślonych bitew. Wszystkie rzeczywiste walki mają jeden lub więcej wystąpień w tym ciągu gdzie aktualna runda jest dopasowana jako rzeczywista a pozostałe są wszystkimi innymi przypadkami (dla $n$ rund walki, istnieje $2^{m-n}$ pasujących elementów). Możemy zobaczyć, że prawdopodobieństwo wymyślonego podzbioru jest równe prawdopodobieństwu rzeczywistej bitwy do której jest dopasowane (ponieważ pierwsze rundy są podobne, kolejne wyimaginowane rundy są niezależne, a możliwość zdarzenia się jakiejkolwiek z dostępnych kombinacji wynosi $1$). Wówczas, każdy ciąg gdzie $k$ lub więcej rund są wygrane oznacza wygranie starcia, a każdy jeden który ma $l$ lub więcej rund przegranych oznacza przegranie starcia (ponieważ $m\leq m_d k + l$, rozważane starcie nie może mieć $l$ przegranych gdy w rzeczywistości ma $k$ wygranych, itd...). Tak więc, powróćmy do modelu Bernoullego: prawdopodobieństwo zwycięstwa jest równe prawdopodobieństwu posiadania $k$ lub więcej sukcesów w ciągu $B(m,p)$. Może to zostać to wyrażone za pomocą przyrostowej funkcji rozkładu Bernoulli-ego (możliwość uzyskania k lub mniej sukcesów):
\begin{equation*}
    F_{B \ N,p}(k) = \sum^{k}_{i=0}
    \binom{N}{i}p^{i}
    (1-p)^{N-1}
\end{equation*}
prawdopodobieństwa wygrania, przegrania i "patu" są uwzględniane \linebreak(jeżeli
 $l\leq m \geq k$)
 \begin{equation}
 \label{eq1}
     P_W = 1 - F_{B \ m,p} (k-1) = F_{B \ m,q}(m-k)
 \end{equation}
 \begin{equation}
 \label{eq2}
     P_L = 1 - F_{B \ m,q} (l-1) = F_{B \ m,p}(m-l)
 \end{equation}
 \begin{equation*}
     P_S = 1 - P_W - P_L = F_{B \ m,p} (k-1) - F_{B \ m,q} (m-l) =
     F_{B \ m,q} (l-1) - F_{B \ m,q} (m-k)
 \end{equation*}
Równoważnie do \ref{eq1}, \ref{eq2} i tych form ( z $m = m_d$) które jest rozwinięte jako
\begin{equation}
    P_W = 1 - \sum^{k-1}_{i=0} \binom{m}{i}p^{i}q^{m-i} = 
    \sum^{m}_{i=k}\binom{m}{i}p^{i}q^{m-i}
\end{equation}
może być przejrzyście ukazane jako przekształcenie algebraiczne, tylko że nikt tego nie zrobił tu jeszcze.
\section{Przybliżone oszacowanie szansy wygranej}
Rozkłady dwumianowe są dla matematyków dobrze przestudiowanymi szczurami laboratoryjnymi, tak więc znane są szybkie sposoby oszacowania ich przybliżenia. Losowe wartości dwumianowe są przedmiotem centralnego twierdzenia granicznego, a $B(N,p)$, z wystarczająco dużym $N$ prowadzi do normalnego rozkładu $N(\mu = N_p, \: \sigma = \sqrt{npq})$ (pierwszy parametr to średnia, drugi to odchylenie standardowe od dwóch rozkładów). Rozkład normalny jest skalowalny od standardowego $N(1,1)$ przy pomocy ciągłego rozkładu prawdopodobieństwa
\begin{equation*}
    \Phi (x) = \int^{x}_{- \infty}
    \frac{e ^{-\frac{x^2}{2}}}
    {\sqrt{2\pi}}\,dx
\end{equation*}
tak, że z funkcją skalowalną $x(\alpha) = \frac{\alpha-\mu}{\sigma}$
\begin{equation*}
    F_{B \: N,p}(k) \approx \Phi(x-(k-0.5))-\Phi(x(-0.5))
\end{equation*}
tak więc
\begin{equation*}
    P_W \approx 1 - \Phi \left(
    \frac{k - mp - 0.5}{\sqrt{mpq}}\right)
    +
    \Phi\left(\frac{-mp-0.5}{\sqrt{mpq}}\right)
\end{equation*}
Skoro $\Phi(x < -3) < 0.0015$ i $\Phi(x > -3) < 0.9985$, możemy być niemalże pewni wyniku bitwy jeżeli wynik $\frac{|mp-k|}{\sqrt{mpq}}$ jest większy niż 3 bez specjalnych obliczeń. W tych przedziałach, funkcja $\Phi(x)$ jest wygładzona i może być łatwo przybliżona. \newline
Problemy zaczynają się gdy $p$ znacznie odbiega od $0.5$, wielkie $N$ dla wiarygodnej wartości przybliżenia staje się bardzo duże.
Górny szacunek błędu ma postać
\begin{equation*}
    \Delta_{N_{max}} = 0.7655 \frac{p^2+q^2}{\sqrt{mpq}} \overset{p \ll 0.5}{\approx} \frac{0.7655}{\sqrt{mp}}
\end{equation*}
Na szczęście dla małych wartości $p$ działa przybliżanie rozkładem Poisson-a $P(\lambda = N_p)$, których parametry równe są obu średnim i rozproszeniom; dla wielu możliwości wystarczy zamienić $p$ z $q$. Funkcja ma postać
\begin{equation}
    F_{P\: \lambda} (k) = \sum^{k}_{i=0} \frac{\lambda^{k}e^{-\lambda}}{k!}
\end{equation}
\begin{equation*}
    P_{W} \overset{p \ll 0.5}{\approx} 1 - F_{P \: mp} (k)
\end{equation*}
Błąd jest ograniczony przez $\Delta_{P_{max}} = \min (mp^{2}, \:p)$ . Rozkład Poisson-a nie może być skalowalny od pojedynczej krzywej, ale sam rząd stabilizuje się wystarczająco szybko. Możemy zasugerować które przybliżenie jest lepsze poprzez przestudiowanie $\frac{\Delta_{N_{\max}}}{\Delta_{P_{\max}}}$. To jak dla $m > 1$, tak, że kiedy ta wartość jest większa niż 2.3, rozkład normalny jest lepszy, a gdy poniżej 1.125, rozkład Poisson-a, a w przedziale pomiędzy tymi liczbami dobrze jest wziąć średnią z obydwu, co da nam dobre oszacowanie (zwłaszcza gdy liczba możliwości jest mniejsza niż $50\%$. ponieważ w ty regionie przybliżenie ma tendencję do błędnych wyników w przeciwnym kierunku), ale ten pomysł wymaga lepszych założeń matematycznych. Rozważana dokładność tego wszystkiego jest w okolicy $\pm 4 \%$  wartości bezwzględnej, i może dochodzić do $\pm 0.5\%$ w przedziale dla $m > 10$.
\section{Odpowiednik zabójstwa}
Każdy atakujący jest scharakteryzowany przez trzy parametry -- zdrowie $a_{hp}$, siłę $s$ i siłę ognia $a_{fp}$ (podobnie ma każdy obrońca; miejmy tutaj na uwadze $m = k+l-1$). Dwóch atakujących z czego co najmniej jeden inny parametr nie mogący atakować celu w ten sam sposób (udowodnij samodzielnie, że intuicyjne założenie "dwukrotnie więcej $f_{p}$ z dwukrotnie mniejszą siłą da prawdopodobnie mniejsze obrażenie" jest \emph{fałszem}. Możemy rozważyć na określonego obrońcę: jaką wartość zdrowia powinien mieć atakujący o standardowej sile i jaką siłę ognia by zabić obrońcę z takim samym prawdopodobieństwem dla atakującego i obrońcy. Dla tego przypadku, rozważmy uproszczone normalne przybliżenie (racjonalną precyzję dla dużego $mq$):
\begin{equation*}
    P_{W} = 1 - P_{L} = F_{B \: m,q}\left(l-1\right) \approx \Phi \left(\frac{l-mq-1}{\sqrt{mpq}}\right)
\end{equation*}
Argument $\Phi$ mówi nam, że $x$ powinien być taki sam. Zatem, mamy nowe $l$ z równania gdzie obliczyliśmy $x$ dla prawdziwego atakującego i wzięliśmy $d_{hp}$ aktualnego obrońcy, i wzięliśmy inne parametry dla naszego standardowego atakującego
\begin{equation*}
    \frac{l-\left(k+l-1\right)q}{\sqrt{\left(k+l-1\right)pq}} = x
\end{equation*}
\begin{equation*}
    \frac{\left(\left(l-1\right)p-kq\right)^2}{\left(k+l-1\right)pq} = x^2,
\end{equation*}
pamiętaj, że $sign((l-1)p-kq)=sign(x)$
\begin{equation*}
    \left(l-1-\frac{kq}{p}\right)^2=x^2\left(\frac{kq}{p}+\left(l-1\right)\frac{q}{p}\right)
\end{equation*}
\begin{equation*}
    \left(l-1\right)^2 -2\left(\frac{kq}{p}+\frac{x^2}{2}\cdot\frac{q}{p}\right)\left(l-1\right)+\left(\frac{kq}{p}\right)^2-\frac{kq}{p}x^2 = 0
\end{equation*}
\begin{equation*}
    l-1 = \frac{kq}{p} + \frac{x^2}{2} \cdot \frac{p}{q} \pm 
    \sqrt{\frac{x^4}{4} \left(\frac{p}{q}\right)^2 + \frac{kq}{p}x^2\frac{p}{q} + 
    \cancel{\left(\frac{kq}{p}\right)^2} - \cancel{\left(\frac{kq}{p}\right)^2} +
    \frac{kq}{p}x^2} = 
\end{equation*}
\begin{equation*}
    \frac{kq}{p} + x\frac{q}{p}\left(\frac{x}{2}\pm\sqrt{
    \left(\frac{x}{2}\right)^2 + \frac{kq}{p}\left(
    \left(\frac{q}{p}\right)^{-1} + \left(\frac{q}{p}\right)^{-2}\right)
    }\right)
\end{equation*}
Możemy łatwo zobaczyć (tylko przenosząc $\frac{k}{p}$ na lewą stronę i zrozumieć, że wyrażenie w nawiasach jest zawsze pozytywne albo negatywne), tak, że ujemne rozwiązanie jest fałszywe, więc
\begin{equation*}
    l = 1+\frac{q}{k}+x\left(\frac{q}{p}\cdot\frac{x}{2}+\sqrt{
    \left(\frac{q}{p}\cdot\frac{x}{2}\right)^2 +k\left(\left(\frac{q}{p}\right)^2+1\right)
    }\right),
\end{equation*}
lub podstawiając właściwości standardowej jednostki
\begin{equation*}
    l_{s, \:a_{fp}}(x)= 1 + \frac{r}{s}\left\lceil\frac{d_{hp}}{a_{fp}}\right\rceil+x\left(
    \frac{xr}{2s} +\sqrt{
    \left(\frac{xr}{2s}\right)^2 + \left\lceil\frac{d_{hp}}{a_{fp}}\right\rceil\left(\left(\frac{r}{s}\right)^2+1\right)
    }\right).
\end{equation*}
Tą wartość prawdopodobnie trzeba będzie zaokrąglić do liczby całkowitej i wprowadzić pewne poprawki iteracyjne do szacowania błędów, ale to pójdzie tak szybko jak szacowanie. Dla przykładu, by powiedzieć w przybliżeniu ilu wojowników ($a_{fp} = 1, s = 1, a_{hp} = 1$), jest równoważnych jednej pół-hit point Artylerii ($a_{fp} = 2, s = 10, a_{hp} = 10$) w ataku na elitarnych Partyzantów ($d_{fp} = 1, r = 1, d_{hp} = 20$), obliczymy
\begin{equation*}
    k = \left\lceil\frac{20}{2}\right\rceil = 10,\: l = \left\lceil\frac{10}{1}\right\rceil = 10, \: \frac{r}{s} = 0.8
\end{equation*}
\begin{equation*}
    x = \frac{10 - 1 - 10 \cdot 0.8}{\sqrt{(10-1+10)\cdot 0.8}} = 0.256,
\end{equation*}
\begin{equation*}
l_{1,1}=1+8\cdot20+0.256\cdot(\frac{0.256\cdot8}{2}+\sqrt{
(\frac{0.256\cdot8}{2})^2+20\cdot(8^2+1)}) = 170.51 \approx 171.
\end{equation*}
Wartość precyzji $P_W$ (obliczana przy pomocy formuły Libreoffice Calc / Excel BINOM.DIST z $q=\frac{8}{8+10}=0.444$ dla artylerii i $q=\frac{8}{8+1}=0.889$ dla Wojowników) są równe 0.689 i 0.635. Prawdziwie, najlepiej oszacowana liczba Wojowników wartych zastąpienia Artylerii w tym przypadku wynosi 171 ($P_W=0.690$).
Możemy użyć tej samej formuły by obliczyć ilość zdrowia (oczywiście, zdrowie atakującego z takimi samymi $f_p$ i siłą $s$ są dodawane z zaokrągleniem do $d_fp$, na przykład k są dodawane) dla każdej określonej jednostki, która zostanie wysłana by zdobyć określony cel, z podanym prawdopodobieństwem. Do obliczenia tego użyjemy funkcji odwróconej normalnej dystrybucji(NORM.S.INV) by uzyskać $x=\Phi^{-1}(P_W)$:
\begin{equation*}
    n=\left\lceil l_{s,a_\text{fp}}|_{\Phi^{-1}(P_W)} \over \left\lceil \frac{a_\text{hp}}{d_\text{fp}}\right\rceil\right\rceil,\quad a_\text{hp}=d_\text{fp}\cdot l_{s,a_\text{fp}}|_{\Phi^{-1}(P_W)}
.\end{equation*}
W powyższym przykładzie, przy użyciu tej funkcji da nam $x=0.492$, $l_{1,1}=179.7\approx180$ i $P_{W}|l_{1,1}=0.715$ co jest nieco bardziej precyzyjne ale wolniejsze do obliczenia. Zamiennie zabicie atakującego może być definiowane w ten sposób także dla jednostek z ograniczoną liczbą rund, ale wyniki tego działania są tak różne, tak, że ma to mało sensu.
\newline Zapamiętaj dwie rzeczy:
\begin{enumerate}
    \item Potrzebujemy znać zdrowie obrońcy (i inne parametry) by przetworzyć zdrowie atakującego.
    \item Celowo użyliśmy elitarnych jednostek w powyższym przykładzie, by uniknąć radzenia sobie z jednostkami o statusie weterana (występują wówczas dodatkowe modyfikatory). 
\end{enumerate}

\section{Przybliżenie z wykorzystaniem klasyfikacji}
Kod sztucznej inteligencji lubi porównywać siłę jednostek względem (nieokreślone, porada) oceny ich ataku/obrony $a_{r} = a_{hp} \cdot s a_{fp},\: d_{r} = d_{hp} \cdot r \cdot d_{fp}$. Czasami kod sugeruje, że $P_{W}=\frac{a^2_r}{a^2_r+d^2_r}$ a taka kwadratowa ocena jednostek oddziałujących wzajemnie (bez wpływu zasady killstack) są dodawane. Ogólnie nie jest to idealnie precyzyjne ale ma trochę sensu. Używając silnego normalnego przybliżenia, otrzymamy to
\begin{equation*}
    P_{W}=\Phi\left(\frac{l_p-kq}{\sqrt{\left(k+l\right)pg}} \right)=
    \Phi\left(\frac{a_r-d_r}{\sqrt{a_{fp}d_{fp}\left(a_rr+d_rs\right)}}\right)\approx
    \Phi\left( \frac{\frac{a_r}{d_r}-1}{\sqrt{\frac{a_r}{d_r}\left(
    \frac{1}{k}+\frac{1}{l}
    \right)}
    }\right)
\end{equation*}
Ta funkcja występuję również w postaci $\frac{
\left( \frac{a_r}{d_r} \right)^2
}{\left( \frac{a_r}{d_r} \right)^2 +1}$
gdy $\frac{1}{k}+\frac{1}{l} \approx 0.79$ ale tylko podobnie gdy oznaczone wartości znacznie się różnią.
Suma kwadratów ocen ma tylko taki sens, że wagi silniejszych jednostek są większe niż suma liniowych ocen, tak więc widzisz, że nie tylko jeden stos jest w stanie pobić inny, ale to jak dużo jednostek które przetrwają bitwę, jest aktualnie pożądanym wynikiem. W rzeczywistości, jeżeli odchylenie jest na prawdę małe, w bitwie 1 jednostki przeciw $n$ jednostką z taką samą sumą ocen zazwyczaj jedna jednostka z którejś strony przetrwa, ale dla jednej strony jedna jednostka to jej $100\%$ a dla innej strony to tylko 1 z $n$, więc dla takich bitew musisz wyprodukować około $n_2$ razy więcej mniejszych jednostek by utrzymać się przy życiu.

\section{Pozostałe HP}
Dobrze, może i wygrałeś. Ale czy chcesz aby twoje zwycięstwo było Pyrrusowe? Czy twoja \emph{Piechota Zmechanizowana} po pokonaniu \emph{Muszkieterów} ma wystarczająco dużo pozostałego \emph{HP} by obronić się w następnej turze przed parą przyczajonych w pobliżu \emph{Wojowników}?
\newline{Nieograniczony dopasowanie wariantów}
Używając powyższych symboli, możemy szybko obliczyć punkty zdrowia atakującego $a_{hp}$ po przegraniu $d$ rund przez broniącą się jednostkę o sile ognia $d_{fp}$
\begin{equation*}
    a'_{hp} = a_{hp} - d \cdot d_{fp}\quad,
\end{equation*}
tutaj $d \in [0;l-1]$. Prawdopodobieństwo posiadania co najmniej $x,x\leq a_{hp}$ pozostałe hp w przypadku zwycięstwa możemy szybko uzyskać modyfikując 2 równanie
\begin{equation*}
    P(a_\text{hp}'\ge x|\text{win}) = \mathrm P\left(d\le \left\lfloor \frac{a_\text{hp}-x}{d_\text{fp}}\right\rfloor=d_x\text{, suppose }d_x\le m|\text{win}\right)=
\end{equation*}
\begin{equation*}
    =\frac{1}{P_W}\sum_{d=0}^{d_x}{k - 1 + d \choose k-1} p^k q^d\text{.}
\end{equation*}
Ta suma wygląda jak \emph{Rozkład Pascala} $NB(k,p)$, to zarządza liczbą $d$ przegranych w rozkładzie Newtona, przed osiągnięciem $k$ sukcesów, co zdarza się z prawdopodobieństwem $p$ każdego przeprowadzonego testu:
\begin{equation*}
    P\left(a'_{hp} \geq x|win\right) = 
    \frac{1}{P_W}F_{NB\: k,p} \left(\left\lfloor
    \frac{a_{hp}-x}{d_{fp}}\right\rfloor\right)
    \text{,}
\end{equation*}
a prawdopodobieństwo przegrania określonej liczby rund $d,0\leq d < l$
\begin{equation*}
    P(d) = f(NB\: k,p)(d) = \dbinom{k-1+d}{d}p^{k}q^{d}
    \text{.}
\end{equation*}
\emph{Rozkład Pascala} oznacza $\frac{kq}{p}=k\frac{r}{s}$. Jeżeli $p\approx 1$ a średnia jest umiarkowana, to można przybliżyć to poprzez \emph{rozkład Poissona} z takimi samymi parametrami; jeżeli $p\approx 0.5$ to jest to zbliżone do normalnego rozkładu z wariancją podobną do niej samej, co oznacza $\sigma^2 = \frac{kq}{p^2}$ (a dla $k=1$ rozkład przyjmuje postać regresji geometrycznej). Ale jest to zdefiniowane do nieskończoności, a to co tu mamy to odcięcie wyników w $l$. Średnia liczba przegranych rund w przypadku wygrania bitwy, zależna od właściwości odciętego rozkładu, będzie następująca
\begin{equation*}
    \langle d\vert d<l \rangle = k\frac{q}{p}-\frac{\frac{kl}{p}f_{NB\:k,p}(l)}{kP_{W}} =
    \frac{kq}{p} - \frac{lf_{NB\:k,p}(l)}{pP_{W}}
\end{equation*}
gdzie nieodcięta funkcja prawdopodobieństwa o określonej wartości
\begin{equation*}
    f_{NB\: k,p}(d)= \binom{k-1+d}{d}p^{k}q^{d}
    \text{.}
\end{equation*}
Podobnie, gdy atakujący zginie, średnia suma obrażeń jakie odniesie obrońca będzie równa
\begin{equation*}
    \langle w\vert w<k \rangle = \frac{lp}{q} - 
    \frac{kf_{NB \: l,q}(k)}{qP_{L}}
    \text{.}
\end{equation*}
Weź pod uwagę, że ekwiwalent zabić nie w ogólnym rozumieniu ekwiwalentem obrażeń. W powyższym przykładzie, jeżeli \emph{Artyleria} zginie, to są redukowane hp \emph{Partyzantów} (nie zapomnij by pomnożyć przez $a_{fp}$) ze średnią na poziomie 13.75, podczas gdy hordy  171, 177 i 180  \emph{Wojowników} zrobią to odpowiednio na 16.4 16.56 i 16.64 zostawiając jakby o połowę mniej.
\newline
Jeżeli użyjemy pojęcia wymyślonej bitwy, wygranie po przegraniu $ d \leq d_{x}$ rund jest równoznaczne wygraniu $k$ lub więcej rund w $k+d_{x}$ bitwie zatem możemy wyrazić skumulowany rozkład obrażeń poprzez dodatnią  funkcję rozkładu Newtona z przesuwalnymi parametrami
\begin{equation*}
    P(a'_{hp}\geq x \vert win)=
    \frac{1}{P_{W}}\sum^{k+d_{x}}_{i=k}\binom{k+d_{x}}{i}p^{i}q^{k+d_{x}-i}=
\end{equation*}
\begin{equation}
    = \frac{1}{P_{W}} ( 1- F_{B \: k+d_{x},p}(k-1))
    \text{.}
\end{equation}
\begin{thebibliography}{99} 
\bibitem{Freeciv}
https://freeciv.fandom.com/wiki/Math\_of\_Freeciv/Battle\_outcome
\end{thebibliography}
\end{document}
